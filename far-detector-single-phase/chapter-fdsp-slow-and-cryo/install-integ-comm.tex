%%%%%%%%%%%%%%%%%%%%%%%%%%%%%%%%%%%%%%%%%%%%%%%%%%%%%%%%%%%%%%%%%%%%
\section{Installation, Integration and Commissioning}
\label{sec:fdsp-slow-cryo-install}
% anselmo, sowjanya

Installation of internal cryogenic pipes should occur soon after the cryostat is completed.
Then, the installation of cryogenics instrumentation devices inside the cryostat will be done in several phases:
\begin{enumerate}
\item Before detector installation: Individual temperature sensors anchored to the cryogenic pipes at the bottom of the cryostat and static T-gradient monitors
  will be installed right after the installation of the pipes. Additional level meters could be also installed during this phase. 
\item During detector installation: Temperature sensors anchored to the top ground planes will be installed in several steps.
  In principle as soon as a CPA enters the cryostat and stored there, the corresponding cables and sensors can be mounted on its ground planes. Cables exceeding the dimensions of
  the ground planes will be roled and stored temperarilly there until they are put in their final position. At this moment cables will be routed towards their ports.   
\item After detector installation: dynamic T-gradient monitors and purity monitors will be deployed  once the detector in installed. 
\end{enumerate}

Since purity monitors will be most likely not hang from a flange they will have to be installed at an earlier stange ... 


Special treatment need the long devices as the purity monitor system, and T-gradient monitors.

The installation of gas analyzers will be inst


%Integration: Initial integration between the two systems will be accomplished at vertical slice and DAQ test stands:
%test of cameras for the detection of sparks, testing SC monitoring/control of HV PS, testing HV interlock status bits, etc.
%Integration/installation of CISC devices (cameras, RTDs) on ground planes will be tested at system integration/assembly sites. Final integration will take place as the two systems are commissioned.


%%%%%%%%%%%%%%%%%%%%%%%%%%%%%%%%%%%%
\subsection{Transport and Handling}
\label{sec:fdsp-slow-cryo-install-transport}

Most instrumentation devices will be shiped to SURF in pieces and mouted onsite. The proper 



%%%%%%%%%%%%%%%%%%%%%%%%%%%%%%%%%%%
\subsection{Integration Facility Operations}
\label{sec:fdsp-slow-cryo-install-facil-ops}


%%%%%%%%%%%%%%%%%%%%%%%%%%%%%%%%%%%
\subsection{Underground Operations}
\label{sec:fdsp-slow-cryo-install-undergr}


%%%%%%%%%%%%%%%%%%%%%%%%%%%%%%%%%%%
\subsection{Integration }
\label{sec:fdsp-slow-cryo-install-integration}

Slow controls needs to be integrated with HV devices as soon as possible 


%%%%%%%%%%%%%%%%%%%%%%%%%%%%%%%%%%%
\subsection{Commissioning}
\label{sec:fdsp-slow-cryo-install-commiss}


Some parts of the CISC system will be commissioned prior to HV commissioning, and before cryostat filling with LAr.
This is the case for RTDs on GPs, cold and inspection cameras and thermal interlocks for PS. Final commissioning of
those systems will be done once the cryostat is filled, since operation will be different in the presence of LAr.
Commissioning the control/monitoring of HV PS and any related hardware interlocks could probably be done at an early
stage as well, provided no real HV is provided to the cathode/field-cages. Final commissioning will be done once HV is
switched on. This will also require coordination from other groups such as LBNF, DAQ, APA etc. The commissioning of the
interfacing elements should follow naturally after (successful) integration testing.

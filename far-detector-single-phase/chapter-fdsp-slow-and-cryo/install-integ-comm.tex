%%%%%%%%%%%%%%%%%%%%%%%%%%%%%%%%%%%%%%%%%%%%%%%%%%%%%%%%%%%%%%%%%%%%
\section{Installation, Integration and Commissioning}
\label{sec:fdsp-slow-cryo-install}
% anselmo, sowjanya

%Then, the installation of cryogenics instrumentation devices inside the cryostat will be done in several phases:
%\begin{enumerate}
%\item Before detector installation: Individual temperature sensors anchored to the cryogenic pipes at the bottom of the cryostat and static T-gradient monitors
%  will be installed right after the installation of the pipes. Additional level meters could be also installed during this phase. 
%\item During detector installation: Temperature sensors anchored to the top ground planes will be installed in several steps.
%  In principle as soon as a CPA enters the cryostat and stored there, the corresponding cables and sensors can be mounted on its ground planes. Cables exceeding the dimensions of
%  the ground planes will be roled and stored temperarilly there until they are put in their final position. At this moment cables will be routed towards their ports.   
%\item After detector installation: dynamic T-gradient monitors and purity monitors will be deployed  once the detector in installed. 
%\end{enumerate}

%Since purity monitors will be most likely not hang from a flange they will have to be installed at an earlier stange ... 


%Special treatment need the long devices as the purity monitor system, and T-gradient monitors.

%The installation of gas analyzers will be inst


%Integration: Initial integration between the two systems will be accomplished at vertical slice and DAQ test stands:
%test of cameras for the detection of sparks, testing SC monitoring/control of HV PS, testing HV interlock status bits, etc.
%Integration/installation of CISC devices (cameras, RTDs) on ground planes will be tested at system integration/assembly sites. Final integration will take place as the two systems are commissioned.


%Some parts of the CISC system will be commissioned prior to HV commissioning, and before cryostat filling with LAr.
%This is the case for RTDs on GPs, cold and inspection cameras and thermal interlocks for PS. Final commissioning of
%those systems will be done once the cryostat is filled, since operation will be different in the presence of LAr.
%Commissioning the control/monitoring of HV PS and any related hardware interlocks could probably be done at an early
%stage as well, provided no real HV is provided to the cathode/field-cages. Final commissioning will be done once HV is
%switched on. This will also require coordination from other groups such as LBNF, DAQ, APA etc. The commissioning of the
%interfacing elements should follow naturally after (successful) integration testing.

\subsection{Cryogenics Internal Piping}
\label{sec:fdsp-slow-cryo-install-pipes}

Installation of internal cryogenic pipes should occur soon after the cryostat is completed.

\subsection{Purity Monitors}
\label{sec:fdsp-slow-cryo-instal-pm}


\subsection{Thermometers}
\label{sec:fdsp-slow-cryo-instal-th}


Individual temperature sensors on pipes and cryostat membrane should be installed prior to any detector component, right after the installation of the pipes.
First, all cable supports will be anchored to pipes. Then each cable will be routed individually starting from sensor end (with IDC-4 female connector but no sensor)
to the corresponding cryostat port. Once all cables going to the same port have been routed, they will be cut to the same length such that they can be properly soldered
to the pins of the SUBD-25 connectors on the flange. In order to avoid damaging the sensors, those will be installed at a later stage just before unfolding the bottom ground planes.

Static T-gradient monitors should be installed before the outer APAs. Thus, the best moment could be right after the installation of the pipes
and before the installation of individual sensors. This will proceed in several steps: i) installation of the two stainless steel strings to the bottom and top corners of the cryostat,
ii) tension and verticality checks, iii) installation of cable supports in one of the strings, iv) installation of sensor supports in the other string, v) cable routing starting from
the sensor end towards the corresponding cryostat port, vi) cut all cables at the same point in that port, vii) solder cable wires to the pins of the SUBD-25 connectors on the flange,
viii) plug sensors onto IDC-4 connectors at a later stage, just before moving corresponding APA insto its final position. 

Individual sensors on top ground plane will have to be integrated with the ground planes. For each CPA (with its corresponding 4 GP modules)
going inside the cryostat, cable and sensor supports will be anchored to the GP threaded rods as soon as possible.
Once the CPA is moved into its final position and its top GPs are ready to be unfolded, sensors on those GPs will be installed. Once unfolded, cables 
exceeding the GP limits can be routed to the corresponding cryostat port either using neighboring GPs or DSS I-bins. 


Dynamic T-gradient monitors will be installed after the completion of the detector ...


\fixme{Missing text for Dynamic T-gradients}

Commissioning of all thermometers will proceed in several steps. Since in a first stage only cables will be installed,
the readout performance and the noise level inside the cryostat will be
tested with precision resistors. Once sensors are installed the entire chain will be check again at room temperature.
The final commissioning phase will be done during and after cryostat filling.  


\subsection{Gas Analyzers}
\label{sec:fdsp-slow-cryo-install-ga}
 

\subsection{Liquid Level Monitoring}
\label{sec:fdsp-slow-cryo-install-llm}


\subsection{Cameras}
\label{sec:fdsp-slow-cryo-install-c}


\subsection{Light emitting system}
\label{sec:fdsp-slow-cryo-install-les}


\subsection{Slow Controls Hardware}
\label{sec:fdsp-slow-cryo-install-sc-hard}



%%%%%%%%%%%%%%%%%%%%%%%%%%%%%%%%%%%%
\subsection{Transport and Handling}
\label{sec:fdsp-slow-cryo-install-transport}

Most instrumentation devices will be shiped to SURF in pieces and mouted onsite. The proper 
Instrumentation devices are in general small except the support structures for Purity Monitors and T-gradient monitors,
which will cover the entire height of the cryostat. Being the load on those structures relatively small (<100 kg) they can be fabricated in parts of less than 3 m,
which can be easily transported to SURF. Those parts can also be movedeasily transported down the shaft and through the tunnels.
All instrumentation devices except the dynamic T-Gradient monitors, which will be introduced into the cryostat through the cryostat port above, can be
moved into the cryostat without the crane.

\fixme{Missing text for  cryogenics internal piping}


%%%%%%%%%%%%%%%%%%%%%%%%%%%%%%%%%%%
%\subsection{Integration Facility Operations}
%\label{sec:fdsp-slow-cryo-install-facil-ops}


%%%%%%%%%%%%%%%%%%%%%%%%%%%%%%%%%%%
%\subsection{Underground Operations}
%\label{sec:fdsp-slow-cryo-install-undergr}


%%%%%%%%%%%%%%%%%%%%%%%%%%%%%%%%%%%
%\subsection{Integration }
%\label{sec:fdsp-slow-cryo-install-integration}

%Slow controls needs to be integrated with HV devices as soon as possible 


%%%%%%%%%%%%%%%%%%%%%%%%%%%%%%%%%%%
%\subsection{Commissioning}
%\label{sec:fdsp-slow-cryo-install-commiss}



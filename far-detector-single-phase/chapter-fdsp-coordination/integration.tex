\subsection{Integration and Systems Engineering }
\label{sec:fdsp-coord-integ-sysengr}
%				7 pages

%%%%%%%%%%%%%%%%%%%%%%%%%%%%%%%%
\subsubsection{Configuration Management}
\label{sec:fdsp-coord-integ-config}

The DUNE Technical Coordination project engineering team will maintain
full 3-D CAD models of the detectors, and the consortia will be
responsible for providing the team with CAD models of their detector
components for integration into the overall models.  The project
engineering team will work with the LBNF project team to integrate the
full detector models into a global LBNF CAD model that includes
cryostats, cryogenic systems, and the conventional facilities.  The
DUNE project engineering team will work directly with the Consortia
Technical Leads and their supporting engineering teams to resolve any
detector component interference and/or connection issues with other
detector systems, detector infrastructure, and facility
infrastructure.

At the time of the release of the Technical Design Reports, the
project engineering team will work with the consortia to produce
formal engineering drawings for all detector components.  These
drawings are expected to be signed by the consortia Technical Leads,
project engineers, and Technical Coordinator.  Starting from that
point, the detector models and drawings will sit under formal change
control.  It is anticipated that designs will undergo further
revisions prior to the start of detector construction, but any changes
made after the release of the Technical Design Reports will need to be
agreed to by all of the drawing signees and an updated, signed drawing
produced.


   (1)	3-D Model
   (2)	Interface Definitions
   (3)	Envelope Drawings for installation
   (4)	Drawing management

%%%%%%%%%%%%%%%%%%%%%%%%%%%%%%%%
\subsubsection{Engineering process and support}
\label{sec:fdsp-coord-integ-engr-proc}
 

The DUNE Technical Coordination organization will work with the
consortia through its project engineering team to ensure the proper
integration of all detector components.  The project engineering team
will document requirements on engineering standards and documentation
that the consortia will need to adhere to in the design process for
the detector components under their responsibility.  Similarly, the
project QA and ES\&H managers will document quality control and safety
criteria that the consortia will be required to follow during the
construction, installation, and commissioning of their detector
components.


Consortia interfaces with the conventional facilities, cryostats, and
cryogenics are managed through the DUNE Technical Coordination
organization.  The project engineering team will work with the
consortia to understand their interfaces to the facilities and then
communicate these globally to the LBNF project team.  For conventional
facilities the types of interfaces to be considered are requirements
for bringing needed detector components down the shaft and through the
underground tunnels to the detector cavern, overall requirements for
power and cooling in the detector caverns, and the requirements on
cable connections from the underground area to the surface.
Interfaces to the cryostat include the number and sizes of the
penetrations on top of the cryostat, required mechanical structures
attaching to the cryostat walls for supporting cables and
instrumentation, and requirements on the global positioning of the
detector within the cryostat.  Cryogenic system interfaces include
requirements on the location of inlet/output ports, requirements on
the monitoring of the liquid argon both inside and outside the
cryostat, and grounding/shielding requirements on piping attached to
the cryostat.

LBNF will be responsible for the design and construction of the
cryostats used to house the detectors.  The consortia are required to
provide input on the location and sizes of the needed penetrations at
the top of the cryostats.  The consortia also need to specify any
mechanical structures to be attached to the cryostat walls for
supporting cables or instrumentation.  The DUNE project engineering
team will work with the LBNF cryostat engineering team to understand
what attached fixturing is possible and iterate with the consortia as
necessary.  The consortia will also work with the project engineering
team through the development of the 3-D CAD model to understand the
overall position of the detector within the cryostat and any issues
associated with the resulting locations of their detector components.

LBNF will be responsible for the cryogenics systems used to purge,
cool, and fill the cryostats.  It will also be responsible for the
system that continually re-circulates liquid argon through filtering
systems to remove impurities.  Any detector requirements on the flow
of liquid within the cryostat should be developed by the consortia and
transmitted to LBNF through the project engineering team.  Similarly,
any requirements on the rate of cool-down or maximum temperature
differential across the cryostat during the cool-down process should
be specified by the consortia and transmitted to the LBNF team.

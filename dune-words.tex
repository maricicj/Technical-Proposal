\documentclass{article}



\usepackage{siunitx}
\input{common/units}

\usepackage{hyperref}

% see dune-words.tex for explanation.

\usepackage[acronyms,toc]{glossaries}
\makeglossaries


% \newduneword{label}{term}{description}
\newcommand{\newduneword}[3]{
    \newglossaryentry{#1}{
        text={#2},
        long={#2},
        name={\glsentrylong{#1}},
        first={\glsentryname{#1}},
        firstplural={\glsentrylong{#1}\glspluralsuffix},
        description={#3}
    }
}

%                 1      2     3       4
% \newduneabbrev{label}{abbrev}{term}{description}
\newcommand{\newduneabbrev}[4]{
  % this makes acronym entries even if they neither term or abbrev is
  % referenced in the text.
  % \newacronym[see={[Glossary:]{#1}}]{abbrev-#1}{#2}{#3}
  \newacronym{abbrev-#1}{#2}{#3}
  \newglossaryentry{#1}{
    text={#2},
    long={#3},
    name={\glsentrylong{#1}{} (\glsentrytext{#1}{})},
    first={\glsentryname{#1}},
    firstplural={\glsentrylong{#1}\glspluralsuffix{} (\glsentrytext{#1}\glspluralsuffix{})},
    description={#4}
  }
}
\newcommand{\dshort}[1]{\acrshort{abbrev-#1}}
\newcommand{\dlong}[1]{\acrlong{abbrev-#1}}

\newcommand{\dword}[1]{\gls{#1}}
\newcommand{\dwords}[1]{\glspl{#1}}
\newcommand{\Dword}[1]{\Gls{#1}}
\newcommand{\Dwords}[1]{\Glspl{#1}}

\newduneabbrev{adc}{ADC}{analog digital converter}{A sampling of a voltage
  resulting in a discrete integer count corresponding in some way to
  the input}

\newduneabbrev{fe}{FE}{front-end}{The front-end refers a point which is
  ``upstream'' of the data flow for a particular subsystem. 
  For example the front-end electronics is where the cold electronics
  meet the sense wires of the TPC and the front-end DAQ is where the
  DAQ meets the output of the electronics}

\newduneabbrev{daq}{DAQ}{data aquisition}{The Data Acquisition system
  accepts data from the detector \acrshort{fe} electronics, buffers
  it, performs a \gls{trigdecision}, builds events from the selected
  data and delivers the result to the offline \gls{diskbuffer}}

\newduneword{detmodule}{detector module}{The entire DUNE far detector is
  segmented into four modules, each with a nominal \SI{10}{\kton}
  fiducial mass}

\newduneword{submodule}{sub-detector}{A detector unit of granularity less
  than one \gls{detmodule} such as the TPC of the single-phase
  \gls{detmodule}}


\newduneword{detunit}{detector unit}{A \gls{submodule} may be partitioned
  into a number of similar parts. 
  For example the single-phase TPC \gls{submodule} is made up of APA
  units}

\newduneabbrev{tpm}{TPM}{trigger primitive message}{A message flowing
  up the trigger hierarchy from local to global context}

\newduneabbrev{tcm}{TCM}{trigger command message}{A message flowing
  down the trigger hierarchy global to local context}

\newduneword{l0primitive}{L0 trigger primitive}{Information derived by
  the DAQ \gls{fe} hardware and which describes a region of space (eg,
  one or several neighboring channels) and time (eg, a contiguous set
  of ADC sample ticks) associated with some activity}

\newduneword{l1primitive}{L1 trigger primitive}{Or \textit{L1 trigger}
  or module trigger. 
  Information derived from \gls{l0primitive} information at the level
  of one \gls{detmodule} and sent to \gls{gtl}}


\newduneabbrev{mtl}{MTL}{Module Trigger Logic}{Trigger processing
  which consumes \gls{detunit} level \gls{l0primitive} information and
  emits \gls{l1primitive} information to the \glspl{gtl}. 
  It also accepts \glspl{trigcommand} from the \gls{gtl} and
  interprets and routes them to \gls{detunit} level components}

\newduneabbrev{gtl}{GTL}{Global Trigger Logic}{Trigger processing
  which consumes \gsl{detmodule} level \gls{l1primitive} information
  and other global sources of trigger input and emits
  \gls{trigcommand} information back to the \glspl{mtl}}

\newduneword{trigcommand}{trigger command}{Information derived from
  one or more \glspl{l0primitive} and which consists of addresses at
  the level of \glspl{detunit} and a period of time. 
  These are delivered to and interpreted by \gls{detunit} level
  components for the purpose of reading out the corresponding data
  from the \gls{ringbuffer}}

\newduneword{ringbuffer}{primary DAQ buffer}{A buffer in the
  \dshort{daq} with sufficient size to store data long enough for a
  trigger decision to be made and with sufficient endurance and
  throughput to allow constant flow of full-stream data}


\newduneword{diskbuffer}{secondary DAQ buffer}{A secondary
  \dshort{daq} buffer holds a small subset of the full rate as
  selected by a \dword{trigcommand}. 
  This buffer also marks the interface with the DUNE Offline}

\newduneword{dumpbuffer}{DAQ dump buffer}{This \dshort{daq} buffer
  accepts a high-rate data stream, in aggregate, from an associated
  \dword{submodule} sufficient to collect all data likely relevant to
  a potential Supernova Burst.}

\newduneword{trigdecision}{trigger decision}{The process by which
  trigger primitives are converted into trigger commands}

\newduneword{octant}{octant}{Any of the eight parts into which 4$\pi$
  is divided by three mutually perpendicular axes. 
  In particular in referencing the value for the mixing angle
  $\theta_{23}$}

\newduneabbrev{dqm}{DQM}{data quality monitoring}{Analysis of the raw
  data to monitor the integrity of the data and the performance of the
  detectors and their electronics. This type of monitoring may be
  performed in real time, within the \dword{daq} system, or in later
  stages of processing, using disk files as input}


\title{DUNE Words}
\author{Brett Viren}

\begin{document}
\maketitle
\tableofcontents

\section{Overview}

The DUNE glossary gives a concise definition of the special terms and
in some cases abbreviations that are part of the DUNE collaboration's
lexicon.
The terms make up a technical vocabulary which DUNE collaborators
use when speaking and writing about the detectors and experiment.

For authoring official documents, particularly ones that span large
parts of the DUNE scope is to \textbf{always} refer to DUNE terms
through a fixed reference and \textbf{never} type them literally into
the body of the document. 
Of course in practice this ideal can only be approached. 
To the text it is, the benefits include:

\begin{itemize}
\item Consistent terms which helps more easily convey meaning to the reader.
\item Reduction in repetitive and potentially inconsistent explanation
  of terms.  
\item Synchronizing of meaning between collaborators.
\end{itemize}

Before writing and during editing a diligent author should find
themselves continuously checking the glossary for previously defined
terms and adding new terms as needed.

The rest of this document describes how to use existing terms from the
glossary when authoring the document body text and how to extend the
glossary to introduce new terms.


\section{Usage}

A \LaTeX{} file \texttt{glossary.tex} defines some DUNE-specific
macros on top of the standard \texttt{glossaries} package followed by
the definitions of the terms themselves. 
This section describes how to use the glossary in a document. 
This document provides an example of its use.

\subsection{Integrating into a document}

To use the DUNE glossary in a document simply \verb|\input| it into
the preamble \textbf{after} the packages \texttt{siunitx} and
\texttt{hyperref}. 
The latter is not strictly required but will allow the resulting PDF
to have clickable glossary links. 

To add a list of used glossary terms to the document add the standard
\verb|\printglossaries| macro provided by the \texttt{glossaries}
package where you want it.

\subsection{Referencing terms}
\label{sec:referencing}

The benefit of the glossary is to use a common vocabulary throughout
the document. 
To do that one should avoid typing a literal DUNE term and instead
reference it through its \textbf{label}. 
Besides assuring consistency it allows editors to easily make sweeping
changes to the ``spelling'' of a term across the document. 
In the case of DUNE terms with abbreviations some automated
conveniences are provided such as including the abbreviation in
parenthesis the first time a term is used and only using the
abbreviation thereafter while in the text always using the same macro.

To reference an item, use one of the following macros. 
The first four cover the case of capitalization and pluralization and,
if an abbreviation exists, will automatically follow the behavior
above. 
To force an abbreviated or long form one the final two may be used.

\begin{itemize}
\item \verb|\dword{label}| nominal term
\item \verb|\dwords{label}| plural term
\item \verb|\Dword{label}| capitalized term
\item \verb|\Dwords{label}| capitalized and plural term
\item \verb|\dshort{label}| force usage of the abbreviated term
\item \verb|\dlong{label}| force usage of the full term
\end{itemize}


The following examples are from this \LaTeX{} source:
\begin{verbatim}
First time, multiple: \dwords{adc}.

Second time, singular: \dword{adc}.

Long: \dlong{adc}.

Short: \dshort{adc}.

First time, singular: \dword{daq}.

Second time, plual: \dwords{daq}.

First time, singular: \dword{detmodule}

Second time, capitalized, plural: \Dwords{detmodule}
\end{verbatim}

First time, multiple: \dwords{adc}.

Second time, singular: \dword{adc}.

Long: \dlong{adc}.

Short: \dshort{adc}.

First time, singular: \dword{daq}.

Second time, plual: \dwords{daq}.

First time, singular: \dword{detmodule}

Second time, capitalized, plural: \Dwords{detmodule}



\section{Extending}

As of this writing, the \texttt{glossary.tex} file is included with a
larger document (currently the DUNE Technical Proposal). 
For future documents some effort to break out this glossary into an
independently managed file should be pursued. 
Until that happens, it is expected that this file will evolve along
and as part of each major DUNE document.

\subsection{Adding new terms}

In general, new DUNE terms may be added to the DUNE glossary using the
macros provided by the standard \texttt{glossaries} package. 
To simplify defining new terms two DUNE-specific macros are defined in
terms of the standard ones.

\noindent To define a DUNE term that has no abbreviation use:

\begin{verbatim}
\newduneword{label}{term}{description}
\end{verbatim}

\noindent To define a DUNE term with an abbreviation use:

\begin{verbatim}
\newduneabbrev{label}{abbrev}{term}{description}
\end{verbatim}

\noindent The labeled arguments are defined as:

\begin{description}
\item[\texttt{label}] a descriptor that will be used to refer to this
  term in the macros described in Section~\ref{sec:referencing}. 
  A good choice for a label for a term with an abbreviation is to
  lowercase that abbreviation. 
  For terms that lack one, something short but similar to the full
  term should be used. 
  The \texttt{label} may contain spaces but making it concise and
  memorable is better than verbose.
\item[\texttt{abbrev}] the abbreviation or acronym for the term (only for \verb|\newduneabbrev|).
\item[\texttt{term}] the DUNE term itself written out in words.
\item[\texttt{description}] a concise but descriptive explanation of
  what the term means. 
  Avoid over specifying and over generalizing. 
  Shoot for one or two sentences. 
  One quirk is that the description must not end in punctuation. 
\end{description}

The term shown in the examples of Section~\ref{sec:referencing} are
defined like:

\begin{verbatim}
\newduneabbrev{adc}{ADC}{Analog Digital Converter}{A sampling of a voltage
  resulting in a discrete integer count corresponding in some way to
  the input}
\newduneabbrev{daq}{DAQ}{data aquisition}{The Data Acquisition system
  accepts data from the detector \acrshort{fe} electronics, buffers
  it, performs a \gls{trigdecision}, builds events from the selected
  data and delivers the result to the offline \gls{diskbuffer}}
\newduneword{detmodule}{detector module}{The entire DUNE far detector is
  segmented into four modules, each with a nominal \SI{10}{\kton}
  fiducial mass}
\end{verbatim}

This latter one gives an example of why \texttt{siunitx} is required. 
Also note that it is acceptable to use \texttt{glossaries} macros to
reference DUNE terms inside the descriptions of other DUNE terms. 
Because the final glossary below only includes terms that have been
referenced, this is a good way to assure completeness.
 



\cleardoublepage
\printglossaries

\end{document}

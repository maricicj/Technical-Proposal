%%%%%%%%%%%%%%%%%%%%%%%%%%%%%%%%%%%%%%%%%%%%%%%%%%%%%%
%
% Where we describe the data volumes that are associated with
% different types of physics programs and their impact on the computing
%%%%%%%%%%%%%%%%%%%%%%%%%%%%%%%%%%%%%%%%%%%%%%%%%%%%%%
\chapter{Data Volume } \fixme{5 pages}
\section{Detector Data Volumes}
\subsection{Far Detector}
\subsection{Near Detector}
\subsection{ProtoDUNE Detectors}

%%%%%%%%%%%%%%%%%%%%%%%%%%%%%%%%%%%%%%%%%%%%%%%%%%%%%%
\section{Beam Physics}
\subsection{Physics Topic Data Volumes}
\subsection{Long Baseline Oscillations}
\subsection{Short Baseline Oscillations}
\subsection{Neutrino Cross Sections}

%%%%%%%%%%%%%%%%%%%%%%%%%%%%%%%%%%%%%%%%%%%%%%%%%%%%%%
\section{Non-Beam Physics}
\subsection{Proton Decay}
\subsection{Astrophysical Observations}
\subsection{Other non-Beam Physics}

%%%%%%%%%%%%%%%%%%%%%%%%%%%%%%%%%%%%%%%%%%%%%%%%%%%%%%
\section{Data Volumes of Supporting Systems}
\subsection{Detector Alignment Systems}
\subsection{Detector Calibration Systems}
\subsection{Detector Performance and Monitoring Systems}

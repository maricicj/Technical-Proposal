%%%%%%%%%%%%%%%%%%%%%%%%%%%%%%%%%%%
\subsection{Cameras}
\label{sec:fdgen-slow-cryo-cameras}
% glenn, jim s, chuck
% same text in single and dual phase

Cameras provide direct visual information about the state of the
detector during critical operations and when damage or unusual
conditions are suspected.  Cameras in the WA105 \(3\times 1\times
  1\SI{1}{m^3}\) dual phase cryostat allowed spray from cool-down nozzles
to be seen, and the level and state of the liquid argon to be observed
as it covered the CRP \cite{Murphy:20170516}.  A camera was used in
the Liquid Argon Purity Demonstrator cryostat\cite{Adamowski:2014daa}
to study high voltage discharges in liquid argon, and in EXO-100
during operation of a TPC \cite{Delaquis:2013hva}.  Warm cameras
viewing LAr from a distance have been used to observe high voltage
discharges in liquid argon in fine detail \cite{Auger:2015xlo}.
Cameras are commonly used during calibration source deployment in
many experiments (e.g., \cite{Banks:2014hra}).

In DUNE, cameras will be used to verify the stability, straightness,
and alignment of the hanging TPC structures during cool-down and
filling; to ensure that there is no bubbling near the ground planes
(single phase) or charge readout planes (dual phase); to inspect the
state of movable parts in the detector (calibration devices, dynamic
thermometers) as needed; and to closely inspect parts of the TPC as
necessary following any seismic activity or other unanticipated
occurence.  These functions will be performed using set of fixed
``cold'' cameras permanently mounted at fixed points in the cryostat
for use during filling and commissioning, and a movable, replaceable
``warm'' inspection camera that can be deployed through any free
instrumentation flange at any time throughout the life of the
experiment.  Table \ref{tab:fdgen-cameras-req} summarizes the
requirements for the camera system.

\begin{dunetable}
[Camera system Requirements]
{p{0.45\linewidth}p{0.50\linewidth}}
{tab:fdgen-cameras-req}
{Camera system requirements}   
 Requirement & Physics Requirement Driver \\ \toprowrule
{\bf Cold cameras} \\ \colhline
minimal heat dissipation when camera not in operation & do not generate bubbles when HV is on \\ \colhline
longevity exceeds 18 months & cameras must function throughout cryostat filling and detector commissioning \\ \colhline
Frame rate \(\geq\SI{10}{\per s}\) & observe bubbling, waves, detritus, etc. \\ \colhline
{\bf Inspection cameras} \\ \colhline
low heat transfer to LAr when in operation & do not generate bubbles, some use cases may require operation when HV is on \\ \colhline
deployable without exposing LAr to air & keep LAr free N2 and other electronegative contaminants \\ \colhline
replaceable camera enclosure & replace broken camera, or upgrade, throughout life of experiment \\ \colhline
{\bf Light emitting system} \\ \colhline
no emitted wavelength shorter than \(\SI{400}{nm}\) & avoid damaging TPB waveshifter \\ \colhline
longevity exceeds 18 months & lighting for fixed cameras must function throughout cryostat filling and detector commissioning \\ \colhline
\end{dunetable}


The following sections describe the design considerations for the cold
and warm cameras and the associated lighting system.  The same basic
design may be used for both the single and dual phase detectors.



% % % %
\subsubsection{Cryogenic Cameras (cold)}

The fixed cameras will be used to monitor the following during filling:
\begin{itemize}
\item positions of corners of APA, CPA, FC, GP (~ 1 mm scale);
\item relative straightness and alignment of APA, CPA, and FC (<~ 1 mm);
\item relative position of profiles and endcaps (~ 0.5 mm)
\item state of LAr surface: is there bubbling? significant? detritus?
\end{itemize}

There are published articles and unpublished presentations describing
completely or partially successful operation of low-cost,
off-the-shelf CMOS cameras in custom enclosures immersed in cryogens.
({\em E.g.,}, EXO-100: \cite{Delaquis:2013hva}; DUNE 35-ton test
\cite{McConkey:2016spe}; WA-105: \cite{Murphy:20170516}.)  Generally
it is reported that such cameras show poor performance and ultimately
fail to function below some temperature of order \(150\sim\SI{
  200}{K}\), but some report that their cameras recover fully after
being stored (not operated) at temperatures as low as \SI{77}{K} and
then brought up to minimum operating temperature.

However, as with photon sensors, experience has also shown that it is
non-trivial to ensure reliable and reproducible mechanical and
electrical integrity of such cameras in the cryogenic environment.
({\em E.g.}, \cite{McConkey:2016spe} and
\cite{Valencia-Rodriquez:20180130}.)  Off-the-shelf cameras and camera
components are generally only specified by the vendors and original
manufactures for operation down to \SI{-40}{\celsius} or \SI{-50}{\celsius}.
In addition, many low-cost cameras use digital interfaces not intended
for long distances, such as USB (\(2\sim\SI{5}{m}\)) or CSI (circuit
board scale), leading to signal degradation and noise problems.

The design for the DUNE fixed cameras will use an enclosure based on
the successful EXO-100 design\cite{Delaquis:2013hva} which was also
used successfully in LAPD. The enclosure will be connected to
stainless steel gas line to allow the enclosure to be flushed with
argon gas at low enough pressure to prevent liquification, using the
same design of the gas line for the beam plug tested in the 35-ton HV
test and in ProtoDUNE.  A thermocouple in the enclosure will allow
temperature monitoring, and a heating element will provide temperature
control.  The camera will transmit its video signal using either a
composite video signal over shielded coax or ethernet over optical
fiber.  Most importantly the DUNE CISC Consortium must work with
vendors to design camera circuit boards that are robust and reliable
in the cryogenic environment.



% % % %
\subsubsection{Inspection Cameras (warm)}


\fixme{turn into complete sentences .... }

Targets: * Status of profiles, endcaps (~ 0.5 mm)
* Status of HV feedthrough and cup
* Observe any y-axis deployment of sources
* Status of thermometers, especially dynamic thermometers
* observe HV discharge or corona on HV FT, cup, or FC

Considerations: need operate only as long as inspection lasts, camera
replacable in case of failure, can keep warm continuously during
deployment and have more options for commercial cameras (e.g., could
deploy same model camera used successfully for HV spark observations
in \cite{Auger:2015xlo}).

Design: enclosure as for cold, mounted on movable/steerable fork, adjustable
height over ~1m range using design similar to dynamic temperature probe.
(Need figure from Jim S!)
Deploy through
gate valve with same purging system as deployable calibration source.
Preserve option for longer deployment with glovebox-based system similar
to KamLAND full-volume calibration system.

% % % %
\subsubsection{Light emitting system}
%%% same text as dual-phase
The light emitting system will be based on Light-Emitting Diodes
(LEDs), with the capability of illuminating interior with selected
wavelengths (IR and visible) that are suitable for detection by the
cameras.  Performance criteria for the light emission system are based
on the efficiency of detection with the cameras, in conjunction with
adding minimal heat to the cryostat. The use of very high efficiency
LEDs will assist with the goal of reducing heat generation; as an
exmple, one \SI{750}{nm} red LED has a specification of
32\%\ conversion of electrical input power to light.

While data on the performance of LEDs at cryogenic temperatures is sparse,
there are some studies related to NASA projects\cite{Carron:2017zzz}, which
indicate that LED efficiency increases with reduced temperature,
and that the emitted wavelengths may change, particularly for ``blue'' LEDs,
but the wavelength changes cited would have no impact on illumination.

\begin{dunefigure}[LED chain for illumination]{fig:gen-cisc-LEDs}
  {Suggested LED chain for lighting inside the cryostat, with
    dual-wavelength and failure-tolerant operation.}
\includegraphics[width=0.7\textwidth]{CISC-Lighting}
\end{dunefigure}

A ``chain'' of LEDs should be connected in series and driven with a
constant-current circuit. It would be advantageous to pair each
LED in parallel with an opposite polarity LED and a resistor
(see Fig.~\ref{fig:gen-cisc-LEDs}).
This allows two different wavelengths of illumination with a single installed
chain (by changing the direction of the drive current) and 
continued use of an LED chain even if individual LEDs have failed.
Installation of multiple LED chains will also mitigate device failures,
and will be necessary for full illumination of the detector.
